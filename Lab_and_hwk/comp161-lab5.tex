\documentclass[]{tufte-handout}
\usepackage{amsmath, amssymb, amsthm}
\usepackage{hyperref}
\usepackage{framed}

\title{COMP161 - Lab 5 \& Homework 4}
\author{}
\date{Spring 2015}

\begin{document}
\maketitle

\begin{abstract}
For this lab and homework you'll develop a function for itemized data and then write two main programs that allow you to test it interactively as describe in Lecture Notes 7.
\end{abstract}

\section{The Problem}

Consider the following adaptation of an HtDP1e problem\sidenote{4.4.4 from \url{http://htdp.org/2003-09-26/Book/curriculum-Z-H-7.html\#node_sec_4.4}}:
\begin{framed}
\begin{quote}
An equation is a claim about numbers; a quadratic equation is a special kind of equation. All quadratic equations (in one variable) have the following general shape:
\[ ax^2+bx+c=0 \]
In a specific equation, $a$, $b$ and $c$ are replaced by numbers, as in
\[ 2x^2+4x+2 = 0 \]
or
\[ 1x^2+0x+(-1) = 0 \]
The variable $x$ represents the unknown.

Depending on the value of $x$, the two sides of the equation evaluate to the same value. If the two sides are equal, the claim is true; otherwise it is false. A number that makes the claim true is a solution. The first equation has one solution, $-1$, as we can easily check:
\[ 2(-1)^2+4(-1)+2 = 0 \]
The second equation has two solutions: $+1$ and $-1$.

The number of solutions for a quadratic equation depends on the values of $a$, $b$, and $c$. Assuming $a$ is not $0$, the equation has
\begin{enumerate}
\item two solutions if $b^2 > 4ac$,

\item one solution if $b^2 = 4ac$

\item no solution if $b^2 < 4ac$
\end{enumerate}

When $a$ is $0$, then the equation is no longer quadratic, it is linear and has the form
\[ bx + c = 0 \]
Linear equations have a single solution so long as $b$ is not $0$. When $b$ is zero, then the equation isn't really an equation but an assertion that is either true when $c$ is $0$ or false otherwise. Alternatively, we could say it has $1$ ``solution'' when $c$ is 0 and $0$ solutions otherwise.  

All three forms of equations are special cases of polynomial equations. Quadratic equations are called second degree polynomials because the largest $x$ term with a non-zero coefficient is squared where linear equations are first degree polynomials because the largest term with a non-zero coefficient is raised to the first power. The final case, the one with no $x$ term, can be re-written as
\[ c*x^0 = 0 \]
From this perspective we see that it's a zeroth degree polynomial.  
\end{quote}
\end{framed}

Your lab and homework revolve around the problem of determining the number of solutions to a polynomial equation of the form $ax^2+bx+c=0$ for any real-valued \sidenote{can be written in decimal notation} $a$, $b$, and $c$. 

\section{Lab 5}

For lab you need to design and develop the function \textit{how\_many}\sidenote{or \textit{howMany} if you prefer camel case} which returns the number of solutions to the polynomial equation with coefficients $a$, $b$, and $c$ as described above. Your function should be developed in the same manner we've been using for the past two weeks: put it in an appropriately named namespace in an appropriately named library and write an appropriate set of gTest unit tests for it.  When you've completed the development and testing of your function, submit it via handin as assignment \textit{lab4}.

\section{Homework 4}

\begin{framed}
\begin{quote}
\textbf{Due by class time on Monday 2/16}
\end{quote}
\end{framed}

For homework you need to develop two main programs in the style shown in lecture notes 7. The first program should be an interactive REPL and the second a one-shot CLI-based program. Both programs should report the results to match the following examples exactly.  If you input 2 for $a$, 4 for $b$ and $2$ for c, then your program should report:
\begin{verbatim}
2(x^2) + 4x + 2 = 0 has 2 solutions
\end{verbatim}
This output should end with a newline such that further I/O doesn't begin after ``solutions''.

Note that this assignment has three executable programs to it: the library unit tests, the main with the REPL and the main that takes arguments from the CLI. You must adapt your Makefile to compile each executable separately. Use the example code found at /home/comp161/sp15/ln7.zip as a guide.  Submit you source code and Makefile via handin as assignment \textit{hwk4}.

\end{document}