\documentclass[]{tufte-handout}
\usepackage{amsmath, amssymb, amsthm}
\usepackage{hyperref}
\usepackage{framed}

\title{COMP161 - Lab 5 \& Homework 5}
\author{}
\date{Spring 2016}

\begin{document}
\maketitle

\begin{abstract}
For this lab you'll get some practice working with the string class and using class objects and methods. 
\end{abstract}

\section{ A function for strings }

The following functional procedure let's you continue to practice the basics while requiring that you use some key methods from the string library. 
\begin{quote}
Design and develop the function \textit{shorten} which takes a string as input and for strings with length greater than 10, it returns a string containing the first 5 characters and the last 5 characters joined by ellipses. For example, the string "this is a long string" would get shortened to "this ...string" and "abcdefghijklmnopqrstuvwxyz" would get shortened to "abcde...vwxyz". If the string's length is less than 10, then it is returned as is from \textit{shorten}. 
\end{quote}
In addition to developing a set of unit tests for this procedure, write a main procedure with a basic REPL for testing this procedure. Instead of using \textit{cin} and the input operators, use the string library procedure \textit{getline} to read in an entire line of text to a string\sidenote{\url{http://www.cplusplus.com/reference/string/string/getline/}}. The shortened version of that line should be reported back to the CLI by your program.

When you're done, or at the end of the lab period, submit your source documents and Makefile using \textit{handin}. The assignment designation is obviously \textit{lab5}. If you didn't complete the program during lab, then complete it and submit the finished code as \textit{hwk5} by the start of class on Friday 2/12.

\section{ More practice with strings and new material? }

If you want to play around with string methods more, you should consider writing up some tests like those seen in lecture notes 8. I highly recommend you play with some of the mutation based methods and writing tests to verify expected mutation effects as this is a new phenomenon.

\end{document}