\documentclass[]{tufte-handout}
\usepackage{amsmath, amssymb, amsthm}
\usepackage{hyperref}


\title{COMP161 - Lab2}
\author{}
\date{Spring 2014}

\begin{document}
\maketitle

\begin{abstract}
For this lab lab you'll explore some basic Bash scripting by beginning Shotts' shell scripting tutorial\sidenote{\url{http://linuxcommand.org/lc3_writing_shell_scripts.php}}.  Shell scripting won't come up on a quiz, but it will teach you a lot about the shell as well as give you some concrete things to do with emacs and the CLI.  
\end{abstract}

\section{Some Motivation and Perspective}

Shell scripts can automate repetitive tasks carried out at the command-line and thereby be a huge timer saver.  As we get in to C++ programming, you'll find that each lab\sidenote{and programming task really} begins with a pretty repetitive sequence of actions:
\begin{enumerate}
\item Make a folder to keep your files in
\item Create ``starter'' versions of anywhere from one to four files.
\end{enumerate} 
The starter files are likely to look more or less the same every time. The folder itself will probably just be named after the assignment.  The key thing here is this: these tasks are repetitive and easily scripted with some basic shell scripting.  So, as you work through Shotts' tutorial, keep this in mind, and as the semester progresses, don't be afraid to take a crack at writing a shell script for tasks you find yourself carrying out on a regular basis.

\section{Lab setup and submission}

This lab\sidenote{and all future labs most likely} will involve several files.  In order to submit all these files with \textit{handin}, \textit{you must put them all in a folder.}  So, before you get started:
\begin{itemize}
\item Create a folder named \textit{lab2} in your home directory and change to that folder.  Do all your work for this lab in that folder.  
\end{itemize}


When the lab is over, you can simply use \textsc{handin} to submit that folder like this:
\begin{enumerate}
\item Return to your home directory
\item Run the command\sidenote{Note that the first \textit{lab2} is the assignment name and the second is the name of the folder being submitted.}: \textit{handin comp161 lab2 lab2}
\end{enumerate}

\section{The Lab}

Follow along with Shotts' shell scripting tutorial\sidenote{\url{http://linuxcommand.org/lc3_writing_shell_scripts.php}}. \textbf{When lab time is up, submit whatever you have}.  You do not need to complete the tutorial but are certainly encouraged to do so. The real goal this week is to \textit{get more practice with Emacs and the CLI as we move towards programming with C++}. 


\end{document}