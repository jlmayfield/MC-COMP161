\documentclass[10pt]{article}
\usepackage{amsmath}
\usepackage{setspace}
\usepackage{hyperref}
\usepackage{listings}
\usepackage{color}

\definecolor{dkgreen}{rgb}{0,0.6,0}
\definecolor{gray}{rgb}{0.5,0.5,0.5}
\definecolor{mauve}{rgb}{0.58,0,0.82}

\lstset{
  %frame=tb,
  language=C++,
  aboveskip=3mm,
  belowskip=3mm,
  showstringspaces=false,
  columns=flexible,
  basicstyle={\small\ttfamily},
  numbers=none,
  numberstyle=\tiny\color{gray},
  keywordstyle=\color{blue},
  commentstyle=\color{dkgreen},
  stringstyle=\color{mauve},
  breaklines=true,
  breakatwhitespace=true,
  tabsize=3
}



\setlength{\textheight}{9in} \setlength{\topmargin}{-.5in}
\setlength{\textwidth}{6.5in} \setlength{\oddsidemargin}{0in}
\setlength{\evensidemargin}{0in}

\title{COMP161 - Lab 10}
\author{ }
\date{Spring 2019}

\begin{document}
\maketitle
\thispagestyle{empty}

\begin{abstract}
For these problems you should simply cram each loop into a single main procedures. No libraries.
No tests. Just a one file program containing a main.
\end{abstract}

\begin{enumerate}
  \item Warm-up: Counting the interval $\left[{0,n}\right)$
      \begin{enumerate}
        \item Write a \textit{for} loop that counts out the interval $[0,n)$ while and computing and finally printing the sum of all the numbers in that interval.
        \item Write a \textit{do-while} loop that counts and computes the sum of the interval $[0,n)$.  Will this loop work exactly the same as your for loop for every value of $n$?
        \item Write a \textit{while} loop that counts backwards through the interval. Have it print the numbers, separated by spaces, as it counts.
      \end{enumerate}
    
  \item In plain English, describe what this loop does. (Hint: Step through it for some small value of $n$ and see exactly what it does for that $n$. Then describe that in more general terms. \textit{Do not translate the code verbatim to English.}). Check yourself by copying and running the code.

    \begin{lstlisting}
      for(int i{0}; i < n; i++ ){
        std::cout << n-1-i;
        if( i % 5 == 4 ){
          std::cout << '\n';
        }
        else{
          std::cout << ' ';
        }
      }
      if( n % 5 != 0 ){
        std::cout << '\n';
      }
    \end{lstlisting}

\item Write a validation loop that is suitable for getting a double from the interval $(0,1)$.

\item Write a loop that counts down through the first $n$ multiples of $3$ and computes their product. When it's done, print that product.

\item Write a loop that prints out every other string in a vector of strings.  Print one string per line. Your loop should continue to work if you change the vector's size, i.e. don't hard-code 10 into the loop if the test vector contains 10 strings.

\item Write a loop that works with a vector of integers and prints out the sum of adjacent, non-overlapping pairs. Assume the vector contains an even number of integers. For example, if the vector contains $\{1,2,3,4,5,6\}$ then it should print $3$, then $7$, and finally $11$ by adding $1$ and $2$, then $3$ and $4$, and finally $5$ and $6$. (Hint: You need to count through the pairs, not the individual vector elements).

\end{enumerate}

\end{document}
