\documentclass[]{tufte-handout}
\usepackage{amsmath, amssymb, amsthm}
\usepackage{hyperref}
\usepackage{framed}

\title{COMP161 - Lab 9 and Homework 6}
\author{}
\date{Spring 2015}

\begin{document}
\maketitle

\begin{abstract}
For this lab and homework you'll exercise your iterative design skills on some vector problems.
\end{abstract}

\section{Choose your own adventure}

Be sure to loop for opportunities to use the pass by const reference optimization when the input vector is read-only. 

\begin{enumerate}
\item \textit{isSorted} \newline
This predicate takes a vector of integers and returns true if the contents of the vector are in least to greatest sorted order and false otherwise. 

\item \textit{find} a.k.a. \textit{search} \newline
Given a vector of integers and a target number called the \textit{key}, find and return the location of the first occurrence of the key or if the key is not in the vector return $-1$. So, if a vector contains $\{0,2,4,6,8\}$ and the key is $4$, you should return $2$. If for that same vector the target was $1$, you'd return $-1$.  

\item \textit{max} \newline
Find and return the maximum value in a vector of doubles. 

\item \textit{minLoc} \newline
Find and return the location of the first occurrence of the minimum value in a vector of doubles. 

\item \textit{mapToOdds} \newline
Compute the odd numbers corresponding to the contents of a vector of integers. The $k$th odd number is $2k+1$. So the number $4$ maps to $2*4+1 = 9$.  Using \textit{mapToOdds} on a vector containing $\{2,4,6,8\}$ should return the vector containing $\{5,9,13,17\}$.  

\item \textit{removeOdds} \newline
Mutate a vector of integers my removing all the odd numbers. Given a vector containing $\{1,2,3,4,5,6\}, \textit{removeOdds}$ should modify the contents such that they're now the vector containing $\{2,4,6\}$. You'll want to explore the vector class documentation\sidenote{\url{http://www.cplusplus.com/reference/vector/vector/}} to see what methods are available for adding and removing elements from vectors.

\end{enumerate}  

\section{Lab 9 Assignment}

Begin work on the problems and when lab is over submit what you have as \textit{lab9} using handin. Do the problems one at a time, not as a batch. I don't want to see declarations and tests for everything with no implementation. I'd rather see one of the problems fully done and nothing for the others. Place all of the procedures in one library and one namespace. 

\section{Homework 6 Assignment}

Complete the problem set and submit the complete set \textit{by class time next Friday}. Submissions should be done via handin. The assignment is, of course, \textit{hwk6}.

\end{document}