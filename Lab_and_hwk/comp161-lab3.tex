\documentclass[]{tufte-handout}
\usepackage{amsmath, amssymb, amsthm}
\usepackage{hyperref}
\usepackage{framed}

\title{COMP161 - Lab 3 \& Homework 3}
\author{}
\date{Spring 2015}

\begin{document}
\maketitle

\begin{abstract}
Lab 3 and Homework 3 give you practice with writing and testing basic functional procedures.  
\end{abstract}

\section{Lab 3}

For lab you'll be working with the following problem from the first edition of \textit{How to Design Programs}.
\begin{framed}
\begin{quote}
\textbf{Exercise 3.3.1.}   The United States uses the English system of (length) measurements. The rest of the world uses the metric system. So, people who travel abroad and companies that trade with foreign partners often need to convert English measurements to metric ones and vice versa.

Here is a table that shows the six major units of length measurements of the English system:


\begin{center}
\begin{tabular}{lcl}
English & &  metric \\ \hline
1 inch	&=&	2.54	cm \\
1 foot	&=&	12	in. \\
1 yard	&=&	3	ft. \\
1 rod	&=&	5(1/2)	yd.\\
1 furlong	&=&	40	rd.\\
1 mile	&=&	8	fl.
\end{tabular}
\end{center}

Develop the functions inches$\rightarrow$cm, feet$\rightarrow$inches, yards$\rightarrow$feet, rods$\rightarrow$yards, furlongs$\rightarrow$rods, and miles$\rightarrow$furlongs.
Then develop the functions feet$\rightarrow$cm, yards$\rightarrow$cm, rods$\rightarrow$inches, and miles$\rightarrow$feet.
Hint: Reuse functions as much as possible. Use variable definitions to specify constants.
\end{quote}
\end{framed}\sidenote{You cannot use $->$ in function names in C++, so you'll need to adjust the names accordingly.}


This problem lists several functions. \textbf{For lab you must complete and submit the inches to centimeters conversion}. Submit your source documents\sidenote{cpp and h files} as assignment \textit{lab3} with \textit{handin}.  Do not submit objects, executables, temporary files created by emacs, or any other non-source file. 

\section{Homework 3}

\begin{center}
\textbf{\textsc{Due by class time Wednesday 2/4}. Submit as hwk3.}
\end{center}

\begin{enumerate}
\item Complete the remainder of the functions from the lab problem for unit conversions.
\item Design and develop a procedure for the following problem\sidenote{again, courtesy of \textit{HtDP1e}}
\end{enumerate}

\begin{framed}
\begin{quote}
\textbf{Exercise 4.4.3.}   Some credit card companies pay back a small portion of the charges a customer makes over a year. One company returns

.25\% for the first \$500 of charges,

.50\% for the next \$1000 (that is, the portion between \$500 and \$1500),

.75\% for the next \$1000 (that is, the portion between \$1500 and \$2500),

and 1.0\% for everything above \$2500.

Thus, a customer who charges \$400 a year receives \$1.00, which is 0.25 * 1/100 * 400, and one who charges \$1,400 a year receives \$5.75, which is 1.25 = 0.25 * 1/100 * 500 for the first \$500 and 0.50 * 1/100 * 900 = 4.50 for the next \$900.

Determine by hand the pay-backs for a customer who charged \$2000 and one who charged \$2600.

Define the function pay-back, which consumes a charge amount and computes the corresponding pay-back amount.
\end{quote}
\end{framed}


\subsection{Problem Set}

These two problems are probably not enough practice with working the design recipe in C++.  Give some serious consideration to working on these problems. You never know, they should show up on a quiz...

Some of these problems might involve Racket strings or Racket symbols. C++ does not have a \textit{symbol} type.  I recommend you use a \textit{char} type instead.  The limitation you'll need to work with is that a \textit{char} value is a single letter, so you'll have to make due with short\sidenote{terrible} names.  Alternatively, you can get ahead of the game and look at the C++ \textit{string} type\sidenote{\url{http://www.cplusplus.com/reference/string/string/}}
\begin{enumerate}
\item From HtDP1e. Problems are listed as \textit{chapter.section.problem} numbers.\sidenote{\url{http://htdp.org/2003-09-26/Book/} }
\begin{enumerate}
\item (Atomics) 3.3.2 - 3.3.5
\item (Itemized) 4.4.4
\end{enumerate}
\item HtDP Online Problem Sets\sidenote{\url{http://htdp.org/2003-09-26/Problems/}}
\begin{enumerate}
\item (Itemized) Section 4
\end{enumerate}
\end{enumerate}

\end{document}