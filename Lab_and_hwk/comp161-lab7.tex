\documentclass[nobib]{tufte-handout}
\usepackage{amsmath, amssymb, amsthm}
\usepackage{hyperref}
\usepackage{framed}

\title{COMP161 - Lab 7 \& Homework 7}
\author{}
\date{Spring 2016}

\begin{document}
\maketitle

\begin{abstract}
For this lab and homework you'll complete the program we started in class (see Lecture Notes 10).
\end{abstract}

\section{Lab 7}

In the course home directory there is a file called \textit{lab7-starter.zip}. It contains the code from lecture notes 7 that you need to complete this lab. In the library header\sidenote{move\_lib.h} you'll see lots of TODOs for documenting and declaring procedures teased out during the design process for our game. These are the \textit{bottom level} procedures of our initial design because we do not imagine them calling or needing any helpers other than perhaps some standard library code. For lab you must:
\begin{enumerate}
\item Complete all the documentation and declaration of the bottom level procedures
\item Stub out all the bottom level procedures
\item Write tests for all the bottom level procedures
\end{enumerate}
When these tasks are complete submit your work as \textit{lab7} using handin. If your implementations are anything more than the most basic of stubs, you'll lose points.

\section{Homework 7}

\begin{center}
\textbf{Due before class on Monday 3/20}
\end{center}

Complete the implementation of your procedures and the complete the program by completing the helpers for main using some combination of your bottom level procedures. The lecture notes (and our in class discussions) discuss the basic options for the implementation of the main helpers. Submit the source for the completed program as assignment \textit{hwk7}.

\end{document}
