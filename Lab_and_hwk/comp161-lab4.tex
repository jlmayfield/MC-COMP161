\documentclass[]{tufte-handout}
\usepackage{amsmath, amssymb, amsthm}
\usepackage{hyperref}
\usepackage{framed}

\title{COMP161 - Lab 4}
\author{}
\date{Spring 2015}

\begin{document}
\maketitle

\begin{abstract}
Today's lab gives you practice with predicate functions, booleans, characters, and libraries.
\end{abstract}

\section{A Character Predicate: \textit{isVowel}}

The C \textit{character type} library\sidenote{aka \textit{cctype} \url{http://www.cplusplus.com/reference/cctype/}} provides several useful predicates that check which variant of the character type you're dealing with. Take a moment to scan the library and see what's available. Your task today is create a new predicate named \textit{isVowel} that determines if a given character is an English vowel\sidenote{A,E,I,O,U,Y} or not. It should work for both uppercase and lowercase letters. \textit{Do not assume that the character in question is a letter.} Focus on getting it working correctly, then see if you can shorten it down and compress the logic to a single return statement with no conditionals\sidenote{assuming that you didn't do this the first go around}.

Just a few reminders:
\begin{itemize}
\item Remember to put \textit{\#include <cctype>} in any file where a cctype library function is invoked.

\item All characters have positive integer values associated with them\sidenote{see \url{http://www.asciitable.com/}}. The C ctype library was written expressly for ints, not chars, so all the cctype library functions take integer arguments and return integer results. The compiler will convert between \textit{int} and \textit{char} types as needed without trouble so you can pretend all those ints are chars.
\end{itemize}

At the end of lab or when you're done with \textit{isVowel}, whichever comes first, submit your source code and Makefile with \textit{handin}. The assignment designation is, of course, \textit{lab4}.

\subsection{Practice/Challenge}

Write an \textit{isConsonant}\sidenote{Non-Vowel letters} function. Leverage your existing \textit{isVowel} function if possible. Also, consider working problems for the problem set given with last week's lab.

\end{document}