\documentclass[]{tufte-handout}
\usepackage{amsmath, amssymb, amsthm}
\usepackage{hyperref}
\usepackage{framed}

\title{COMP161 - Lab 4}
\author{}
\date{Spring 2014}

\begin{document}
\maketitle

\begin{abstract}
For this lab you'll practice basic procedure design for atomic and itemized data in preparation for Monday's Quiz.
\end{abstract}

\section{Problem Set}

Develop procedures for any of the following problems. 
\begin{enumerate}
\item From HtDP1e. Problems are listed as \textit{chapter.section.problem} numbers.\sidenote{\url{http://htdp.org/2003-09-26/Book/} }
\begin{enumerate}
\item (Atomics) 3.3.2 - 3.3.5
\item (Itemized) 4.4.4
\end{enumerate}
\item HtDP Online Problem Sets\sidenote{\url{http://htdp.org/2003-09-26/Problems/}}
\begin{enumerate}
\item (Itemized) Section 4
\end{enumerate}
\end{enumerate}
 
Remember: Document+Declare $\rightarrow$ Write Tests $\rightarrow$ Stubs $\rightarrow$ Code $\rightarrow$ Run Tests. Any of the first four steps are fair game for the quiz. So understand each of them separately and how they work together in the course of procedure design and development. You can put all your procedures in a single library if you want\sidenote{call it something like lab5}.  Be sure to include the problem number in the documentation of each procedure. For problems from the problem set, label them \textit{PS.4.x} where $x$ is the problem number. At the end of lab, submit your source code as assignment \textit{lab5} using \textit{handin}.

\end{document}