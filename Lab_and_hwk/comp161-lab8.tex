\documentclass[nobib]{tufte-handout}
\usepackage{amsmath, amssymb, amsthm}
\usepackage{hyperref}
\usepackage{framed}

\title{COMP161 - Lab 8 \& Homework 8}
\author{}
\date{Spring 2016}

\begin{document}
\maketitle

\begin{abstract}
In this lab you'll work on recursive and iterative procedures for strings.
\end{abstract}

\section{The Problem: Removing Letters}

For this lab you'll work on a functional solution to the problem of removing all the letters from a string. To make them distinct, put them in different namespaces\sidenote{You've seen me do this in a lot of code}.  

\subsection*{Lab 8}

For lab, you should begin by setting up the documentation, stubs, and tests for \textit{two} versions of this procedure: one iterative and one recursive. Submit this code with handin as assignment \textit{lab8}.  

\subsection*{Homework 8}

\begin{center}
\textbf{Due by class time on Wednesday. Submit as assignment \textit{hwk8}}.
\end{center}

For homework, complete the implementation of both versions of the function.  

\section{Practice Problems}

The following problems are good practice problems for developing your recursive and iterative problem solving skills. 
\begin{itemize}
\item Count Vowels
\item Remove everything but digits
\item Shift letters up so that A becomes B, B becomes C, etc.  The letter Z should wrap around to A. 
\end{itemize}


\end{document}