\documentclass[nobib]{tufte-handout}
\usepackage{amsmath, amssymb, amsthm}
\usepackage{hyperref}
\usepackage{framed}

\title{COMP161 --- Lab 8}
\author{}
\date{Spring 2018}

\begin{document}
\maketitle

\begin{abstract}
In this lab you'll practice basic iterative and recursive design on the classic problem of finding the max. There is no required homework with this lab. It is, however, recommended you finish it out by Friday so that reviewing the solutions is more meaningful for you.
\end{abstract}

\section{Lab 8}

On the server in \textit{/home/comp161/sp18} you'll find c161-lab8-starter.zip. It contains the basic setup for implementing a basic max finding function for a vector of doubles. Get it and unzip it to your home directory. Then:
\begin{enumerate}
\item Put your name in each of the file headers
\item Write tests for both implementations in \textit{lab8-test.cpp}.  These tests can, and should, test the same data. The implementation doesn't change the problem.
\item Finish documentation by filling in the pre and post conditions
\item Implement the iterative version
\item Implement the recursive version
\end{enumerate}

When you're done, or lab is over, submit your lab directory as assignment \textit{lab8} using the handin script.

\end{document}
