\documentclass[]{tufte-handout}
\usepackage{amsmath, amssymb, amsthm}
\usepackage{hyperref}
\usepackage{framed}

\title{COMP161 - Lab 8}
\author{}
\date{Spring 2015}

\begin{document}
\maketitle

\begin{abstract}
In this lab you'll work on recursive and iterative procedures for strings.
\end{abstract}

\section{Remove Letters}

For this lab you'll work on the functional problem of removing all the letters from a string: \textit{given a string compute the string that contains all the non-letter characters in that string}. Begin by setting up the documentation, stubs, and tests for \textit{two} versions of this procedure: one iterative and one recursion. To make them distinct, put them in different namespaces\sidenote{You've seen me do this in a lot of code}.  Both the iterative, loop-based version, and the recursive version must be \textbf{completed by class on Friday}. Submit your solutions as \textit{lab8} via the handin program.

\section{Practice Problems}

The following problems are good practice problems for developing your recursive and iterative problem solving skills. 
\begin{itemize}
\item Count Vowels
\item Remove everything but digits
\item Shift letters up so that A becomes B, B becomes C, etc.  The letter Z should wrap around to A. 
\end{itemize}


\end{document}