\documentclass[]{tufte-handout}
\usepackage{amsmath,amssymb,amsthm}


\title{Comp161 - Lab 1 and Homework 1}
\author{}
\date{ Spring 2014 }

\begin{document}
\maketitle

\begin{abstract}
Your first homework and lab assignments get you moving with the Linux command line interface\sidenote{CLI} and the GNU Emacs text editor.  You'll be using these tools on an almost daily basis for this class and it is vital that you get comfortable with them ASAP. The best way to do this is by doing.  Towards this end you'll be working through two tutorials:
\begin{enumerate}
\item William Shotts, \textit{Learning the Shell} from \url{http://linuxcommand.org}.
\item The Emacs built-in tutorial. You can launch the Emacs tutorial from within Emacs by using the command \textit{C-h t}.  Emacs itself is launched from the CLI with the command \textit{emacs}.
\end{enumerate}
Be sure to run all the commands Shotts shows you as he shows them to you.\sidenote{Some commands won't work because you lack sufficient privileges or the system doesn't support them}
\end{abstract}


\section{Lab 1}

For your first lab you'll need to work a little of both tutorials.  Your goal is to work through some of the  Shotts tutorial and some of the the Emacs tutorial.\sidenote{Be certain you can get in and out of the CLI, EMACS, and the EMACS tutorial so that you can complete your homework}.  \textit{Be certain to review the homework assignment as you'll want to work on that as you progress through the tutorials.} 

By the end of lab you must: \textit{use Emacs to create a file named lab1report.txt and in that file respond to the questions listed below}\sidenote{Don't forget to put your name on the file!}. When you're done \textit{use the \textit{handin} program}\sidenote{see below}\textit{ to submit the lab report text file.}.  Don't go over board with your responses, I just want to learn a bit about your background.
\begin{enumerate}
\item What did you do over Winter break?
\item Have you ever used Linux before? If so, give a brief explanation of your history with Linux.
\item Have you ever worked with the Linux or Windows CLI before? If so, give a brief explanation of your history with the CLI. If not, what is your initial reaction to working with Linux and the CLI?
\end{enumerate}

\section{Homework 1}

\begin{center}
\textbf{Due at the start of lab on Wednesday 1/22}
\end{center}

For homework you'll need to complete the remainder of the Shotts tutorial and the Emacs tutorial.  My way of checking that you've done this is by having you show me two things:

\begin{enumerate}
\item A shell\sidenote{bash really} reference that allows you to quickly remind yourself of the commands covered in Shotts' tutorial. Your free to add anything else you want to this\sidenote{like things from the resources given below}. I count something like 40+ commands in Shotts' tutorial. Some are given to you in passing. Others are discussed at length.  \textit{I'm looking for your reference to have some organization to it\sidenote{Look at the EMACS sheet and how commands are organized}.}

\item The Emacs reference sheet with all the commands discussed in the Emacs Tutorial highlighted.  
\end{enumerate}

Every year there are students that never really learn to work quickly and efficiently with these tools.  They constantly need reminders of even the most basic commands. In the end, they spend a good 20-30 minutes just reminding themselves basic commands and doing things the slow way\sidenote{be on the lookout for auto-completion and command history}. The other common problem is typing too much. Using the CLI right means minimize keystrokes. This, in turn, means reducing the chance you'll make a typo. Typos either cause problems you have to fix or they don't do anything and render the command you just typed meaningless.  Once again, it because a huge waste of your time.  All this wasted time in the lab is you could have been spending working on the real class material and getting help from me if needed. So, you really need to commit to learning these tools, go beyond the assignments and really see what you can do with this stuff\sidenote{I won't force you to do this but I will pester you about it if you don't}.

\section{Emacs}

Emacs is the text editor we'll be learning in this class. We've discussed it what you need for these assignments in class, but here's a quick reminder. Commands usually require you to combine some keys with the \textit{ctrl}\sidenote{shown as \textbf{C} on the sheet} key or the meta key\sidenote{Shown as M. See below.}. For example, the command to close emacs is written \textit{C-x C-c}. That means, ``press and hold ctrl then x, then release them, then press and hold ctrl then x, and release them.''  It should feel like your rolling through keys starting with ctrl.  If you're familiar with the windows command \textit{ctrl-alt-del}, then you know what I'm talking about. 

If you're at the CLI, you need two things really:
\begin{itemize}
\item \textit{To launch emacs}: emacs
\item \textit{To open/create a file with emacs:} emacs \textit{filename}
\end{itemize}

Once you're in emacs you'll need these three emacs commands.
\begin{itemize}
\item \textit{To start the tutorial}: C-h t
\item \textit{Close Emacs:} C-x C-c
\item \textit{Save current file:} C-x C-s
\end{itemize}

The tutorial will break down all the other essentials of emacs.

\subsection{Meta Key}

If you're on a linux or windows machine, then you have an \textit{alt} key. That's your meta key. So commands like \textit{M-b} are telling you to press and hold alt then b, then release both. If, however, you're on a Mac, you lack and alt key. You have two options\sidenote{\url{http://stackoverflow.com/a/3566557/1042494}}: use the \textit{Esc} key or tell your terminal to use \textit{option} as the meta key.  If you go the route of \textit{esc}, then I don't believe you hold the the key down\sidenote{I could be wrong about that.}.

\section{Handin}

The \textit{handin} program is a shell script that deposits files into a directory found at $ /home/comp161$, and it's how you'll be submitting most of the work for this class\sidenote{you should add it to your list of commands on your reference}. The command \textit{handin -h} displays the help text for handin, which in turn tells you everything you need to know about using it manage the submission of your work. \textit{Read the handin help text to figure out how to submit your lab assignment.} The assignment designation for labs will always be something like \textit{lab1} or \textit{lab7}.  This week is, of course, \textit{lab1}.  The course designation in this class is always \textit{comp161}.  For this lab, you'll only need to submit the one text file, lab1report.txt.  \textbf{You'll eventually want to make a directory for each lab and assignment so that you can submit that entire directory.}\sidenote{keep your work organized or risk misplacing/losing assignments}

\section{Other Sources}

Along the lines of really committing to these tools, there are a few excellent additional resources I want to point out to you. Zed Shaw is a programmer that writes tutorials/online classes.  His CLI course is great if you really just want to drill the commands in to your fingertips.  It's worth checking out and it's free.
\begin{itemize}
\item Shaw, Zed. \textit{The Command Line Crash Course: Controlling Your Computer From The Terminal}. Dec 2011. \url{http://cli.learncodethehardway.org/book/}
\end{itemize}
Additionally, Eric Nodwell's quick tutorial is nice because it really just gets down to the stuff you use most often.
\begin{itemize}
\item Nodwell, Eric. \textit{Introduction to Commandline Linux}. 2003. \url{http://www.phas.ubc.ca/~mbelab/computer/linux-intro/html/}
\end{itemize}
Finally, William Shotts' tutorial continues on to introduce you to shell scripts. It's worth your time.  Furthermore, all his online material is drawn form his book.  The book is free and probably worth downloading or even buying \url{http://linuxcommand.org/tlcl.php}.

The built-in Emacs tutorial is great. But if you want another perspective on Emacs, check this write-up out.
\begin{itemize}
\item Wacelna, Keith. \textit{A Tutorial Introduction to GNU Emacs}. 2009. \url{http://www2.lib.uchicago.edu/keith/tcl-course/emacs-tutorial.html}
\end{itemize}


\end{document}