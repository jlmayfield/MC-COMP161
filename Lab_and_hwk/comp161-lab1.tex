\documentclass[nobib]{tufte-handout}
\usepackage{amsmath,amssymb,amsthm}


\title{Comp161 \\ Lab 1 and Homework 1}
\author{}
\date{ Spring 2017 }

\begin{document}
\maketitle

\begin{abstract}
Your first homework and lab assignments get you moving with the Linux command line interface\sidenote{CLI} and the GNU Emacs text editor.  You'll be using these tools on an almost daily basis for this class and it is vital that you get comfortable with them ASAP\@. The best way to do this is by using them.  \textbf{A lot}. To start out you will work through two tutorials:
\begin{enumerate}
\item William Shotts, \textit{Learning the Shell} from \url{http://linuxcommand.org}.
\item The Emacs built-in tutorial. You can launch the Emacs tutorial from within Emacs by using the command \textit{C-h t}.  Emacs itself is launched from the CLI with the command \textit{emacs}.
\end{enumerate}
Be sure to run all the commands Shotts shows you as he shows them to you.\sidenote{Some commands won't work because you lack sufficient privileges or the system doesn't support them}.  You might also need to refer to lecture notes 2 and 3 for help with submitting the assignments, the basics of Emacs, and other useful discussion about the CLI\@.
\end{abstract}


\section{Lab 1}

For lab you must work through some of the  Shotts tutorial and some of the the Emacs tutorial.\sidenote{Be certain you can get in and out of the CLI, EMACS, and the EMACS tutorial so that you can complete your homework outside of lab, and be certain to review the homework assignment before you get going on the lab as you'll want to work on that as you progress through the tutorials.}

By the end of lab you must:
\begin{enumerate}
\item Create a folder in your home directory named \textit{comp161} and in that folder create a folder named \textit{lab1}.
\item Use emacs to create a file named lab1report.txt. This file should be located in the folder \textit{lab1} that you created in the previous step. In the file respond to the following questions\sidenote{Don't forget to put your name on the file!}:
\begin{enumerate}
\item What did you do over Winter break?
\item Have you ever used Linux before? If so, give a brief explanation of your history with Linux.
\item Have you ever worked with the Linux or Windows CLI before? If so, give a brief explanation of your history with the CLI\@. If not, what is your initial reaction to working with Linux and the CLI\@?
\end{enumerate}
Don't go over board with your responses, I just want to learn a bit about your background.
\item Use the \textit{handin} program\sidenote{see Lecture Notes 2}to submit the lab1 folder.
\item Before you leave lab, show the instructor that you created all the files and folders requested above.
\end{enumerate}

\section{Homework 1}

\begin{center}
\textbf{Due in class Wednesday 1/25. Returned in lab that same day.}
\end{center}

For homework you'll need to complete the remainder of the Shotts tutorial and the Emacs tutorial.  My way of checking that you've done this is by having you show me two things:

\begin{enumerate}
\item A shell\sidenote{bash really} reference sheet \textit{of your own making} that allows you to quickly remind yourself of the commands covered in Shotts' tutorial. The sheet should be your own work. You can copy the design of sheets found online, but you must create the sheet from scratch. It may be handwritten or typed. You're free to add anything else you want to this\sidenote{like things from the resources given below}. I count something like 40+ commands in Shotts' tutorial. Some are given to you in passing. Others are discussed at length.  \textit{I'm looking for your reference to have some organization to it\sidenote{I recommend something like this: \url{http://www3.uah.es/clima/staff/gianni/doc/practicas/Extra/bash_reference_sheet.pdf}}. Hastily thrown together notes with little to know organization will likely receive a 1 for the assignment.}

\item The Emacs reference sheet with all the commands discussed in the Emacs Tutorial highlighted. This means you need to go to the course website, find the sources document, follow the link to the Emacs reference sheet, and print the sheet.
\end{enumerate}

You really need to commit to learning these tools, go beyond the assignments and really see what you can do with this stuff\sidenote{I won't force you to do this but I will pester you about it if you don't}. The CLI has systems in place to help you work quickly and efficiently\sidenote{be on the lookout for auto-completion and command history!}. Every year there are students that try to get these tutorials done as fast as possible and don't really dig into the material in them. These students typically never really learn to work quickly and efficiently with these tools and spend much of their time frustrated. So, take your time with the tutorials and build a really usable shell reference. You'll thank yourself later.


\section{Other Sources}

Along the lines of really committing to these tools, there are a few excellent additional resources I want to point out to you. Zed Shaw is a programmer that writes tutorials/online classes.  His CLI course is great if you really just want to drill the commands into your fingertips.  It's worth checking out and it's free.
\begin{itemize}
\item Shaw, Zed. \textit{The Command Line Crash Course: Controlling Your Computer From The Terminal}. Dec 2011. \url{http://cli.learncodethehardway.org/book/}
\end{itemize}
Additionally, Eric Nodwell's quick tutorial is nice because it really just gets down to the stuff you use most often.
\begin{itemize}
\item Nodwell, Eric. \textit{Introduction to Commandline Linux}. 2003. \url{http://www.phas.ubc.ca/~mbelab/computer/linux-intro/html/}
\end{itemize}
Finally, William Shotts' tutorial continues on to introduce you to shell scripts. It's worth your time.  Furthermore, all his online material is drawn form his book.  The book is free and worth downloading or even buying \url{http://linuxcommand.org/tlcl.php}.

The built-in Emacs tutorial is great. But if you want another perspective on Emacs, check out this write-up.
\begin{itemize}
\item Wacelna, Keith. \textit{A Tutorial Introduction to GNU Emacs}. 2009. \url{http://www2.lib.uchicago.edu/keith/tcl-course/emacs-tutorial.html}
\end{itemize}


\end{document}
