\documentclass[nobib]{tufte-handout}
\usepackage{amsmath, amssymb, amsthm}
\usepackage{hyperref}
\usepackage{framed}

\usepackage{listings}
\usepackage{color}

\definecolor{dkgreen}{rgb}{0,0.6,0}
\definecolor{gray}{rgb}{0.5,0.5,0.5}
\definecolor{mauve}{rgb}{0.58,0,0.82}

\lstset{
  language=C++,
  aboveskip=3mm,
  belowskip=3mm,
  showstringspaces=false,
  columns=flexible,
  basicstyle={\small\ttfamily},
  numbers=none,
  numberstyle=\tiny\color{gray},
  keywordstyle=\color{blue},
  commentstyle=\color{dkgreen},
  stringstyle=\color{mauve},
  breaklines=true,
  breakatwhitespace=true,
  tabsize=3
}

\title{COMP161 \\ Loops}
\author{}
\date{Spring 2017}

\begin{document}
\maketitle

\begin{abstract}
  Practice with Loops for the loop Exam.
\end{abstract}

\begin{enumerate}
  \item In plain English, describe what this loop does. (Hint: Step through it for some small value of $n$ and see what it does. Then describe that in general terms. \textit{Do not translate the code verbatim to English.}).

    \begin{lstlisting}
      for(int i{0}; i < n; i++ ){
        std::cout << n-1-i;
        if( i % 5 == 4 ){
          std::cout << '\n';
        }
        else{
          std::cout << ' ';
        }
      }
      if( n % 5 != 0 ){
        std::cout << '\n';
      }
    \end{lstlisting}

  \item Write a loop (any kind) that counts up through the first $n$ multiples of $7$ and prints them each.

  \item Write a loop that counts down through the first $n$ multiples of $3$ and computes their product. When it's done, print that product.

  \item Write a while loop that prints the characters of a string in reverse.

  \item Write a loop that keeps getting integers from the user until they enter a multiple of $11$ at which point it prints that multiple and stops.

  \item Write a validation loop that is suitable for getting a double from the interval $(0,1)$.

  \item Write a guessing game loop. Users will be prompted to guess a number (integer). If their guess is some secrete number $n$ (just pick an $n$) then ``Congratulations!'' is printed to the screen and the loop ends. If it's wrong, they are prompted for a Yes/No answer to the question ``Do you want to guess again?''. (Hint: this is a riff on the until-quit loop that has two ways of quitting).

\end{enumerate}



\end{document}
