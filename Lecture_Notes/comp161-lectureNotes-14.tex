\documentclass[]{tufte-handout}
\usepackage{amsmath,amssymb,amsthm}
\usepackage{listings}
\usepackage{color}

\definecolor{dkgreen}{rgb}{0,0.6,0}
\definecolor{gray}{rgb}{0.5,0.5,0.5}
\definecolor{mauve}{rgb}{0.58,0,0.82}

\lstset{
  frame=tb,
  language=C++,
  aboveskip=3mm,
  belowskip=3mm,
  showstringspaces=false,
  columns=flexible,
  basicstyle={\small\ttfamily},
  numbers=left,
  stepnumber=1,    
  firstnumber=1,
  numbersep=5pt,
  numberfirstline=true,
  numberstyle=\tiny\color{black},
  keywordstyle=\color{blue},
  commentstyle=\color{dkgreen},
  stringstyle=\color{mauve},
  breaklines=true,
  breakatwhitespace=true,
  tabsize=3
}


\title{COMP 161 - Lecture Notes 14 - Search and Sort}
\date{}

\begin{document} 
\maketitle

\begin{abstract}
In these notes we apply the basic recursive and iterative design template to the problems of searching and sorting. 
\end{abstract}

\section{Structural Recursion and \textit{std::vector}s}

Structural recursions is about making recursive procedures that recurse on the recursive structure of the data. The most basic form of this structure come from having an empty collection \textsc{base case} and a non-base case where the data is deconstructed into the first element and all of the rest. The ``rest'' is a smaller version of the original structure. More generally, we can identify any non-recursive smallest size\sidenote{one element, two elements, etc} as a base case and deconstruct the non-base case in any way so long as repetition of the decomposition eventually results in a base case\sidenote{all but the last + last, left half + right half, etc.}. Such structures exists abstractly for just about any collection type.  However, not all collections support recursive decomposition or do so in an inefficient manner. The C++ \textit{std::vector} does not support recursive decomposition directly. 

Thankfully, any structure with indexed elements can be managed recursively thought the recursive handling of the set of index values. A vector of size \textit{s} used the integer interval $[0,s)$ for its indices. When the vector is empty, then the interval $[0,0)$ is also empty\sidenote{$[a,b)$ is empty if $a \geq b$}. The first of this interval is $0$ and the rest is $[1,s)$. Similarly, the last is $s-1$ and all but the last is $[0,s-1)$. We can generalize this for any range of contiguous positive integers $[a,b)$. The interval is empty if $a \geq b$. When $a < b$, the first is $a$ and the rest is $[a+1,b)$.

To recursively process a vector we must write a procedure that takes the index interval bounds $first$ and $last$.  We don't always need both bounds. Pure functions on vectors can often exclude the last when recursing first to last or last when recursing last to first.  Having both, however, provides maximum flexibility.  If you need to mutate the vector and doing so changes the size, then you're probably going to need to shift first and last to account for the change in the vector's structure. You also can design with both bounds so that you can defer choosing the exact pattern of recursion to implementation time. Sometimes you don't realize that a pattern won't work until you really start working with the details. 

If your goal is to work with a vector in its entirety, then the extra parameters change the basic interface to that problem. To hide them we use a \textsc{overloaded} function where one version takes on the vector and calls the recursive procedure with first equal to zero and last equal to the size of the vector. This will cause the recursive variant of the function to consider all the elements of the vector. 

Figure \ref{code:vecrec} gives the basic template for first + rest structural recursion on vectors with a top-level variant that works on the whole vector by using the recursive variant as a helper. We'll be applying this template to search and sort.
\begin{figure}[htpb!]
\begin{lstlisting}
/**
 * 
 *@param v the vector 
 *@param first the smallest index to be processed
 *@param last the excluded upper bound of the interval of indexes to be 
 *  processed
 *@return ...
 *@pre 0 <= first,last < v.size()
 *@post ...
 */ 
... foo([const] std::vector< ... >& v, int first, int last){
 
   if( first >= last ){
      // base case
      ....
   }
   else{
      ... v[first] ... foo(v,first+1,last)
   }
	
}

/**
 * 
 *@param v the vector 
 *@return 
 *@post ...
 */ 
... foo([const] std::vector< ... >& v){
  ... foo(v,0,v.size()) ...	
}

\end{lstlisting}
\caption{Structural Recursion Template for Vectors}
\label{code:vecrec}
\end{figure}


\section{Search: The Problem}

Search is a fundamental problem in computing. Given a collection\sidenote{std::vector for now}, find a specific element, typically called the \textit{search key}\sidenote{or just \textit{key}} in that collection. Several variations can occur: find the first occurrence, the last occurrence, and more generally the $k-th$ occurrence.    With the \textit{std::vector}, we want to return the index of occurrence. A simpler version of search is the \textit{contains} predicate that returns true if the collection contains at least one occurrence of the search key. Finally, another search-like procedure is a counting procedure that returns the total number of occurrences. 

For these notes we'll look at the version of search that examines the entire vector and returns the index of the first occurrence 
of the key. The declaration for this function is given in figure \ref{code:searchdecl} with tests in figure \ref{code:searchtests}.

\begin{figure}[!htpb]
\begin{lstlisting}
/**
 * Compute the location of the first occurrence of the integer key
 * in the vector data.
 * @param data vector of integers
 * @param key search value
 * @return -1 if key is not found, otherwise the index where key
 * is first found
 * @pre none
 * @post none
 */
int search(const std::vector<int>& data,int key);
\end{lstlisting}
\label{code:searchdecl}
\caption{The Basic Search Function}
\end{figure}

\begin{figure}[htpb!]
\begin{lstlisting}
TEST(search,all){

   EXPECT_EQ(-1,search(std::vector<int>({}),1));
   EXPECT_EQ(-1,search(std::vector<int>({2}),1));
   EXPECT_EQ(0,search(std::vector<int>({1}),1));
   EXPECT_EQ(1,search(std::vector<int>({1,3,5}),3));
   EXPECT_EQ(0,search(std::vector<int>({1,3,1}),1));
   EXPECT_EQ(2,search(std::vector<int>({1,3,5}),5));

}
\end{lstlisting}
\label{code:searchtests}
\caption{Tests for Basic Search}
\end{figure}

This basic interface and the tests should work regardless of the underlying implementation technique.  How you solve the problem does not change the problem itself.  In this case, we're interested in searching an entire vector and this function captures that problem succinctly.   

\section{Search: The Solutions}

The recursive and iterative versions share a lot in common. We'll start with the recursive variant and then look at the iterative version. In both cases it's clear that we only need read-only access to the vector and using \textit{pass-by-const-reference} would be a good idea.

\subsection{Recursive Search}

For the recursive implementation we need the variant of the search that accepts the bounds of the search range in order to recurse along that interval as shown in figure \ref{code:searchrecdecl}. It's wroth noting this function, as declared, doesn't need to be recursive. We can, and should, view it as a more general version of search: find the key within this sub-section of the vector. The more general problem of searching a part of a vector gives us the flexibility we need to enable recursion. 

\begin{figure}[htpb!]
\begin{lstlisting}
/**
 * Compute the location of the first occurence of the integer key
 * in the vector data for the index range [fst,lst).
 * @param data vector of integers
 * @param fst the lower bound of the search range
 * @param lst the excluded upper bound of the search range
 * @param key search value
 * @return -1 if key is not found, otherwise the index where key
 * is first found
 * @pre fst <= lst
 * @post none
 */
int search(const std::vector<int>& data,int fst, int lst, int key);	
\end{lstlisting}
\label{code:searchrecdecl}
\caption{Recursive-Capable Search}
\end{figure}

When testing the more generic search we should test it not only for whole vector searches, the problem we originally set out to solve, but for partial vector searches as we see in figure \ref{code:searchrectests}.  Test the procedure as it stands on its own, not just for some subset of its usage. 
\begin{figure}[htpb!]
\begin{lstlisting}
TEST(search,some){

  // search all
  EXPECT_EQ(-1,ln14::search(std::vector<int>({}),0,0,1));
  EXPECT_EQ(1,ln14::search(std::vector<int>({1,2,3,2,1}),0,5,2));
  EXPECT_EQ(-1,ln14::search(std::vector<int>({1,2,3,2,1}),0,5,7));
  // search some
  EXPECT_EQ(3,ln14::search(std::vector<int>({1,2,3,2,1}),2,5,2));
  EXPECT_EQ(-1,ln14::search(std::vector<int>({1,2,3,2,1}),2,3,2));

}
\end{lstlisting}
\label{code:searchrectests}
\caption{Recursive-Capable Search Tests}
\end{figure}

The top-level search\sidenote{search the whole vector} simply calls the more generic version with the interval for the entire vector as shown in figure \ref{code:searchrectop}.
\begin{figure}[htpb!]
\begin{lstlisting}
int search(const std::vector<int>& data,int key){
	return search(data,0,data.size(),key);
}
\end{lstlisting}
\label{code:searchrectop}
\caption{Top-Level Search: Recursive Implementation}
\end{figure}

To work out the recursive implementation of the generalized search we start with the base case. When a vector is empty, it cannot contain the key so return $-1$. With the non-empty case we work out the problem in terms of the first\sidenote{data[fst]} and the result of recursing on the rest\sidenote{search(data,fst+1,lst,key)}. In the context of search this boils down to the following observation: either the first is the key, the key is in the rest, or the key is not in the vector at all. A complete case analysis reveals four possible situations\sidenote{two potential locations (first and rest) with two potential states(contains and doesn't contain)}: the first is the key and the rest contains the key, the first isn't the key and the rest contains the key, the first is the key and the rest doesn't contain the key, or the key is neither the first nor is it in the rest. By recursively calling search on the rest we expect to get the location of key in the rest or a -1 if it's not in the rest.  For example, if the vector $v$ contains $\{2,3,2,1\}$ and the search key is $2$ then the recursive call \textit{search(v,1,v.size(),2)} should return 2, but because $2$ is also at location $0$, the first, we should return $0$ in favor of $2$.  If we were searching for the key $4$ then the recursive call should return $-1$ and given that the first, $2$ isn't $4$, we should return $-1$. By continuing to work examples like this we see that the recursion is necessary if and only if the first isn't the key. In the case where the first element is the search key, we should just return the first index without recursively searching the rest. 

In figure \ref{code:searchrecwork} we see the finished code for the recursive search. The two cases of the non-empty portion of the main conditional have been flattened into the main conditional itself rather than nesting a second conditional in the else of the main conditional. 
\begin{figure}[htpb!]
\begin{lstlisting}
int search(const std::vector<int>& data,int fst, int lst, int key){
	if( fst >= lst ){
		return -1;
	}
	else if( data[fst] == key ){
		return fst;
	}
	else{
		return search(data,fst+1,lst,key);
	}
}
\end{lstlisting}
\label{code:searchrecwork}
\caption{Search Some: Recursive Implementation}
\end{figure}

It's worth stopping to check our implementation against our case analysis. The most important case is the recursive one: the search range of the vector is either empty or it's not. Empty case is handled by the \textit{if} and the non-empty case is caught either by the \textit{else if} or the \textit{else}. When the search range isn't empty we can condense the four cases down to two: the key is the first or its existence and location can be determined through recursively searching the rest. The \textit{else if} case should catch the case where the first is the key. The \textit{else} covers the other case. 

\subsection{Iterative Search}

The basic logic of iteration is to traverse the structure and accumulate the solution while you traverse. Vector traversals work the same way as string taversals. You simply count your way through the set of index values\sidenote{[0,size)}. By default we start at zero and count up, but other counting schemes are possible. The solution to this problem is an index value or -1.  Developing the recursive solution showed us that we shouldn't need to continue searching once we discover the first occurrence of the search key. In an iterative world this means stopping the traversal and returning the current index. Combine this with the fact that we need to accumulate an index value, we notice that an extra accumulator isn't required, the loop is already accumulating what we need in it's counter. We need to be careful though. Basic accumulator logic dictates that we return the accumluated value when traversal is done.  When we don't find the search key the loop's counter will be the size of the vector and we want -1. Similarly, if the vector is empty, the initial counter value is 0 and we need to return -1. The fix here is to recognize that when we find the key, we can return the counter, and if we don't find the key we just return the literal -1. 

In figure \ref{code:searchiter} we see the finished code for the recursive search. The two cases of the non-empty portion of the main conditional have been flattened into the main conditional itself rather than nesting a second conditional in the else of the main conditional. 
\begin{figure}[htpb!]
\begin{lstlisting}
int search(const std::vector<int>& data,int fst, int lst, int key){

    for(unsigned int i{0}; i < data.size() ; ++i){
    	if( data[i] == key ){
			return i;
		}
    }

    return -1;
}
\end{lstlisting}
\label{code:searchiter}
\caption{Search: Iterative Implementation}
\end{figure}

It's worth your time to really analyze this solution and clearly identify how we tweaked the basic ``traverse and accumulate'' logic to arrive at this implementation. Imagine we're searching the vector $v$ containing $\{2,3,1,2\}$ for the search key $1$. If we're currently looking at index $i=1$ then we've accumulated the fact that everything before $1$ doesn't contain the key\sidenote{i.e. our accumulator is -1}. The vector element at $1$ isn't the key, so we should leave the accumulated value as $-1$. If the key were $2$ then we should have discovered the key on a previous iteration and the accumulator would be $0$, the location of the first occurrence of the key in the previously traversed structure. Finally, consider the case where the key is $2$ and the current location is $3$. We should have discovered $2$ at location $0$ on a previous iteration and upon rediscovering $2$ and location $3$ we should preserve the previous accumulator value because our goal is to find the first occurrence. If you wanted to capture this kind of on the nose iterative thinking in code you'd end up with what you see in figure \ref{code:searchiter2}. From this perspective our version is an optimization of the full traversal version\sidenote{why keep going once you've found it and why accumulate the location twice, once for the traversal loop and once for the iteration}. 

\begin{figure}[htpb!]
\begin{lstlisting}
int search(const std::vector<int>& data,int fst, int lst, int key){

	int loc{-1};
    for(unsigned int i{0}; i < data.size() ; ++i){
    	if( data[i] == key && loc == -1 ){
			loc = i;
		}
    }

    return loc;
}
\end{lstlisting}
\label{code:searchiter2}
\caption{Search: Iterative Implementation with explicit accumulation and full traversal}
\end{figure}

\section{Sort: The Problem}

Sorting is something we teach small children before they even learn to read and do arithmetic. The problem is that fundamental to our thinking. It may come as a surprise then that it provides a wide array of variations.  For starters, we can sort in ascending or descending order. As a computing problem we can think of it as a function and a mutator. The the case of the former, we'd be producing a second vector with the same contents as the argument but a potentially different order. As a mutator we would be committing to rearranging the contents of an existing vector. Of course, we could implement a function via mutation of a local vector and we can implement a mutator by making a local sorted copy then swapping the original for the contents of the sorted copy. 

The version of sort we'll be looking at is an ascending order, \textsc{in-place} sort. Working in-place is just another way of saying we want to implement sort as a vector mutator and manage the mutation directly rather than copying the contents and modifying the copy. In-place sorts are important to computing because they guarantee we do not make copies of data and in general, copying can be costly. 

Another desirable property of sorts is \textsc{stability}. A stable sort will leave equivalent values in the same relative order as they were originally\sidenote{the first 1 is still first, the second 1 is still second, and so on}. We'll set aside stability as a design goal for now but should step back from whatever implementation we end up with and determine if we just so happened to produce a stable sort. 

In figure \ref{code:sortdecl} we see the declaration for a sort mutator. The accompanying tests are give in figure \ref{code:sorttests}. When developing tests it's good to work examples in increasing order of complexity. This typically boils down to increasing sizes and maybe different cases for a particular size. For vectors of size zero or one, the sort seems trivial. There's really nothing to do. For a size of two we can get a feel for when work does and does not need to happen because sometime we need to swap the order and sometimes we don't. Finally, vectors of size greater than run a wide gambit of nearly sorted to totally unsorted. 

\begin{figure}[htbp!]
\begin{lstlisting}
/**
 * Sort the contents of data in least to greatest order
 * @param data vector of integers
 * @return none
 * @pre none
 * @post contents of data have been sorted in least to greatest order
 *  for all i in [0,data.size()-1), data[i] <= data[i+1]
 */
void sort(std::vector<int>& data);
\end{lstlisting}
\label{code:sortdecl}
\caption{Sort as a Mutator}
\end{figure}

\begin{figure}[htbp!]
\begin{lstlisting}
TEST(sort,all){

	std::vector<int> sortme;

 	sort(sortme);
    EXPECT_EQ(std::vector<int>({}),
	      sortme);

    sortme = std::vector<int>({1});
	sort(sortme);
    EXPECT_EQ(std::vector<int>({1}),
	      sortme);

    sortme = std::vector<int>({7,4});
	sort(sortme);
    EXPECT_EQ(std::vector<int>({4,7}),
	      sortme);

    std::vector<int> data{8,7,6,5,4,3,2,1};
	sort(data);
    EXPECT_EQ(std::vector<int>({1,2,3,4,5,6,7,8}),
	      data);

}
\end{lstlisting}
\label{code:sorttests}
\caption{Sort as a Mutator}
\end{figure}

This time we'll start by looking at the iterative implementation because loops and iteration plan nice with mutation and perhaps we'll have an easier time teasing out basic logic in that environment. Once we've teased out some iterative sorting logic, we'll see if that doesn't provide insight into a recursive strategy.

\subsection{Iterative Implementation}

Iterative procedures still have bases cases, cases where no iteration\sidenote{traverse and accumulate} is needed. The question we ask ourselves is: when can we tell that a vector is sorted without actually traversing the vector in any way? The answer is when the vector contains one or fewer items. Here we see an instance of a non-empty base case: Vectors of size one need not be worked on because they are trivially sorted already. As we move to implementing our sort we want to ensure that our traversal loop only traverses vectors with size greater than one. 

Iterative mutation is about accumulating effect, not value. If we've done $i$ steps of the sorting traversal than the vector should be sorted in the first $i$ places. This also means that when the loop is done, the sort should be done. There's nothing left to do. We need to combine this mutation based thinking with the base case now.  If the base case includes regions of size one, then our traversal should begin at the second element in the vector and accumulate effect across the whole region of size two. This all pushes us to a loop design that begins at $i=1$ and counts up towards the size of the vector. Beginning at one means the loop will only perform iterations for vectors with a size greater than one\sidenote{assuming we continue as long as $i < size$ of course}.

The last thing we need to work out is how to accumulate. Let's look an example. If the vector $v$ originally contained $\{3,7,1,2,8,6\}$ and we've performed $2$ iterations, then the current index $i$ should be $3$ and $v$ from $0$ to $2$ should contain $\{1,3,7\}$. We're now faced with the problem of modifying the region from $0$ to $3$ such that the $2$ at location $3$ is placed at location $1$ and the $3$ and $7$ following it are shifted over to the right one. It is not immediately clear how to do this. 

Because we're trying to work in-place, it's important that we think in terms of moving data within the existing vector and not using any kind of operation that changes the size of the vector. This rules out the use of the std::vector \textit{insert} and \textit{erase} methods. There is an \textit{assign} method\sidenote{\url{http://www.cplusplus.com/reference/vector/vector/assign/}} that might be useful, but in order to use it we have to know where, exactly, things need to go.  When we look at our example, we see what needs to go where because our visual processing lets us work out the global structures. The computer doesn't see that. It sees nothing. The computer has a sorted region and an element right next to that region that needs to get moved so that the whole region is now sorted. There appears to be no pre-build solution to this\sidenote{or we just want to solve this problem on our own anyway}, so the answer is, of course, to abstract away the problem in the form of a helper. 

When faced with a task for which no known method, procedure, or operation provides an solution you should quickly design a helper to solve the problem. This is the bread and butter of top-down design\sidenote{The real win here is that you're able to solve the problem without actually solving the problem by clearly identifying, naming, and specifying the missing piece(s)}. Find the helpers you need by working them out in context. What we need is a procedure that takes the vector and the bounds of a region of that vector. To shake things up we'll include the upper bound of the region this time. So, our precondition is that for the region contained in $[fst,lst]$, the data in $[fst,lst)$ is sorted. We know nothing about the element at $lst-1$. The procedure should then move the data at $lst-1$ such that everything in $[fst,lst)$ is now sorted. We're also assuming that the sorted region contains at least one item. This means $fst < lst$. In short, this procedure inserts the item at \textit{lst} into the region to the right of \textit{lst} and that that ends at \textit{fst} so that the whole region is sorted. We can now properly declare and document this as a C++ procedure as shown in figure \ref{code:insertiterdecl}.

\begin{figure}[htbp!]
\begin{lstlisting}
/**
 * Move data[lst] into sorted region data[fst..lst-1] such that
 * the whole region is sorted.
 * @param data vector of integers
 * @param fst lowest index of region for insertion
 * @param lst the location of the item to be inserted. Also
 *  one more than the last of the insertion region
 * @pre fst < lst. data[fst .. lst-1] is sorted in
 *  least to greatest order
 * @post data[fst..lst] is sorted in least to greatest order
 */
void insert(std::vector<int>& data,
   		    unsigned int fst, unsigned int lst);
\end{lstlisting}
\caption{Insert: Declaration}
\label{code:insertiterdecl}
\end{figure}

A few tests for insert are given in figure \ref{code:insertitertests}. 	Once again, develop tests ranging from simple base cases to larger more complex cases. We should also test insert in a more general setting. We expect the value of \textit{fst} to always be 0 but it doesn't need to be. Testing more general cases lets you re-evaluate and reaffirm the pre and post conditions. 
\begin{figure}
\begin{lstlisting}
TEST(insert,all){

    std::vector<int> testme({3,2});
    ln14::iter::insert(testme,0,1);
    EXPECT_EQ(std::vector<int>({2,3}),
	      testme);

    testme = std::vector<int>({2,3});
    ln14::iter::insert(testme,0,1);
    EXPECT_EQ(std::vector<int>({2,3}),
	      testme);
   
    testme = std::vector<int>({2,4,5,7,2,4});
    ln14::iter::insert(testme,0,4);
    EXPECT_EQ(std::vector<int>({2,2,4,5,7,4}),testme);

    testme = std::vector<int>({9,4,5,7,2,1});
    ln14::iter::insert(testme,1,4);
    EXPECT_EQ(std::vector<int>({9,2,4,5,7,1}),testme);
    
}
\end{lstlisting}
\label{code:insertitertests}
\caption{Insert: Tests (Iterative Version)}
\end{figure}

Now that it's clear what \textit{insert} should do and how to use it, we can finish up \textit{sort}. In figure \ref{code:sortiter} we see the complete iterative sort known as \textsc{Insertion Sort}. Insertion sort is a class, well known algorithm for sorting algorithm. Our design uses an insert helper. It's not uncommon to see the helper replaced by the actual insertion logic. This design clearly separates the core iterative logic of the sort from the accumulative operation carried out by insert. We're also able to express \textit{sort} without knowing how to insert. In short, this design clearly separates different concerns of the problem when compared to the explicit, all-in-one version you'll find in texts and online. 

\begin{figure}[htbp!]
\begin{lstlisting}
void iter::sort(std::vector<int>& data){

  for(unsigned int i{1}; i < data.size(); i++){
    iter::insert(data,0,i);
  }
  return;
}
\end{lstlisting}
\label{code:sortiter}
\caption{Sort: Iterative Sorting}
\end{figure}

We now need to finish the implementation of \textit{insert} if we want sort to actual work. If our sort is to be an in-place mutator, then \textit{insert} must also work in-place through mutation. Once again, we'll choose to use iteration rather than recursion in order to implement insert. 

The first thing we need to recognize is that \textit{insert} needs to traverse and accumulate through the portion of the vector indexed by $[fst,lst)$. The in-place requirement really pushes us away from the standard fst to lst traversal though. Let's see why. Say some vector $v$ contains $\{1,3,5,7,2\}$ in the $[fst,lst]$ range\sidenote{So v[fst] is 1 and v[lst] is 2}. After a single iteration from fst to lst, what should the vector range look like? What we need to remember is that the region we've traversed should be the final solution with respect to that portion of the vector. So after a single iteration the fst should still be 1. After two iterations we should see $\{1,2,3,5,7\}$ because the 2 needs to be at $fst+1$, the second spot, in the final solution. If we're to do this in place, then we need to shift everything in $[i,lst)$ to the right one so that $v[lst]$ can go at location $i$.  Once again, it seems like we'd need a helper for this because there's no operation for shifting a region. Before we go down that path, let's explore a different traversal patter: lst-1 to fst. 

When traversing starting at lst-1 and working down towards fst we need not only rethink the loop counting pattern but the iterative logic. The region we've traversed already is not at the back of the range, not the front. After one iteration with our same example vector $v$, what should $v$ look like? The basic idea now is that if we're dealing with location $i$ on the current iteration that after this iteration $[i,lst]$ should be sorted and contain all the values that were originally in $[i,lst]$. So after one iteration our vector should look like $\{1,3,5,2,7\}$. What we've actually done is reversed the problem a bit. We want to insert the value at $lst$ into the sorted region $[fst,lst)$.  In working lst to fst, we actually inserting the last of $[fst,i]$ into the sorted region $(i,lst]$. Initially the region $(i,lst]$ is just $lst$. After the first iteration, it's $[lst-1,lst]$, and then on down.  

The fact that both regions are sorted lets us cut some corners. Let's trace insert a bit to find it.After one iteration our example region of $v$ contains $\{1,3,5,2,7\}$. We can accomplish this by swapping the last two elements or more generally $v[i] and v[i+1]$. On the second iteration we need to ``insert $5$ into $\{2,7\}$''.  Once again a simple swap would accomplish this. In fact, if you continue on you'll find that all we ever need to do is swap $v[i]$ with $v[i+1]$ until the original insertion number, $2$, gets to its final location at which point we can just stop traversing all together. Why is this?  Until we get the original $v[lst]$ to the correct spot, $v[i+1]$ is always that number and everything to its right (if it exists) must be greater than or equal to all the stuff in $[fst,i]$ because that makes up the original sorted region $[fst,lst)$! When we swap $v[i]$ with $v[i+1]$ we're simply putting a number back where it was but shift to the right of the original $v[lst]$ rather than the left of $v[lst]$ where it started. When $v[i]$ is less than $v[i+1]$, then we can stop the traversal because we know everything to the left of $v[i]$ (if it exists) is less than or equal to $v[i]$ and must therefore be less than $v[i+1]$. By traversing from right to left rather then left to right reversed the problem a bit but in doing so naturally solved our shift and place problem by turning the shift into iterated swaps. 

The final version of iterative \textit{insert} is shown in figure \ref{code:insertiter}. It's design and implementation merit repeated study as it illustrates how varying the traversal pattern can sometimes avoid some problems encountered with the standard left to right traversal at the cost of forces us to reverse our thinking on the iterative logic of the problem. The other thing you'll see is careful management of counting down through an index range using unsigned integers. If fst is $0$\sidenote{and it always will be for our target usage of insert}, then when the unsigned int $i$ is equal to $fst$ and we do $i--$ we won't get $-1$ but a very large number instead. The check against the size of the vector ensures we never carry out an iteration with an $i$ value outside of $[fst,lst]$.  
\begin{figure}[htpb!]
\begin{lstlisting}
void iter::insert(std::vector<int>& data,
	    		  unsigned int fst, unsigned int lst){

	for(unsigned int i{lst-1}; i >= fst && i < data.size(); i--){
    	if( data[i+1] < data[i] ){
			std::swap(data[i],data[i+1]);
      	}
      	else{
		 return;
        }
    }
    return;    
}
\end{lstlisting}
\label{code:insertiter}
\caption{Insert: Iterative Implementation}
\end{figure}



\subsection{Recursive Implementation}


\end{document}