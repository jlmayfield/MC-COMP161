\documentclass[10pt]{article}
\usepackage{amsmath}
\usepackage{setspace}
\usepackage{hyperref}


\setlength{\textheight}{9in} \setlength{\topmargin}{-.5in}
\setlength{\textwidth}{6.5in} \setlength{\oddsidemargin}{0in}
\setlength{\evensidemargin}{0in}

\title{Syllabus - COMP 161 - Introduction to Programming}
\author{ James \textit{Logan} Mayfield }
\date{Spring 2015}

\begin{document}
\maketitle

\section{Logistics}
\begin{itemize}
\item \textbf{Where: } 
\begin{itemize} 
\item Class: Center for Science and Business (CSB), Room 323	
\item Lab: Center for Science and Business (CSB), Room 309
\end{itemize}
\item \textbf{When: } MWF 8-8:50, (Lab) W 11am - 12:50pm
\item \textbf{Instructor :} James \textit{Logan} Mayfield
\begin{itemize}
\item \textit{Office: } Center for Science and Business (CSB), Room 344
\item \textit{Phone: } 309-457-2200
\item \textit{Email: } lmayfield \textit{at} monmouthcollege \textit{dot} edu
\item \textit{Office Hours: }  By Appointment.
\end{itemize}
\item \textbf{Credits: } 1 course credit
\end{itemize}
\emph{Note: This Syllabus is subject to change based on specific class needs. Significant deviations from the syllabus will be discussed in class.}

\section{Required Text}

There is no formal text for this course. It is taught using the instructor's lecture notes, found at \url{https://github.com/jlmayfield/MC-COMP161}, and freely available resources on the web.  A mostly complete list of these resources is included as an attachment to this syllabus. The lecture notes are very much a work in progress and subject to updates as the semester progresses.  

\section{Programming Environment}

We'll be working with the \textit{C++ programming language} in this course and developing our programs using the \textit{Linux command-line environment}, \textit{GNU GCC compiler}, and the \textit{EMACS} text editor.  Several other \textit{Linux-based tools} will be utilized to debug and evaluate your code throughout the semester. You are expected to develop your work on the department's server, which you will access remotely using \textit{SSH}.  You may choose to do your work on your own machine. However, if you do so, it is your responsibility to properly port your work to the server and submit it properly.  

\textit{All of the tools used in this course are freely available, open-source tools which you may download via the web.}


\subsection{SSH Clients}

In order to access the department's server from your personal machine you need to obtain an SSH client. If you need any help getting an SSH client up and running on your machine, just ask.  If you're running Linux or Apple's OS, then you've already got a CLI SSH client available to you.  If you're on windows, then you can get the same SSH client we'll use in the lab. 
\begin{itemize}
\item \textbf{Windows:} Putty \url{http://www.chiark.greenend.org.uk/~sgtatham/putty/}
\item \textbf{Mac and Linux} You already have a CLI/terminal based client!  
\end{itemize}

\section{Description and Content}

Introduction to Programming, COMP161, is a continuation of the core CS curriculum that began in COMP 160, Fundamentals of Computer Science.  In COMP161, we will turn our attention to developing software systems and the craft of programming.  Today's programmers face many challenges when developing software systems. This course is designed to begin to prepare students to understand and meet those challenges.  One of the more crucial skills a programmer needs is the ability to critically analyze and evaluate programs and, based on their analysis, make qualitative and quantitative judgments about the quality of that program.  

In COMP161, students will hone their program analysis skills through in class practice and hands on lab exercises.  Homework will enforce key ideas and concepts as well as give students practice with basic language syntax and mechanics.  Students will carry out two programming projects.  These projects will require much more work than labs and are designed to get students comfortable working with the complexities of larger-scale software projects.

\subsection{Content}

Topics that will be covered in COMP161:
\begin{itemize}
\item Goals of Programming and Software Development
\item Programming Tool-Chains
\begin{itemize}
\item Text Editors
\item Compilers
\item Debuggers
\item Profilers and Analyzers
\item Version Control Systems
\end{itemize}
\item The Command Line Interface (CLI) 
\item Imperative, Procedural Programming with C++ using:
\begin{itemize}
\item recursive and iterative procedures
\item state variables
\item C++ Templates
\item basic Classes \& Structs
\item C++ STL Vectors and Arrays
\item C++ STL Lists and Linked-Lists
\item CLI and File Input/Output
\end{itemize}
\item Algorithm Analysis and Big-O Notation
\end{itemize}  


\section{Expectations and Policies}

You are expected to carry yourself in a mature and professional manner in this course. Towards this end, there are a few classroom policies by which you are expected to abide.
\begin{itemize}

\item \textit{Late Assignments: } In general, late assignments will \textit{not} be accepted.  If you feel you have a justified reason for the assignment being late you may set up an appointment to meet with the instructor and plead your case.  Situations beyond your control are understandable and exceptions can and will be made. 

\item \textit{Attendance: } \textbf{Repeated absences and late arrivals to class will quickly reduce your participation grade to zero.}  The occasional late arrival or missed class is one thing, but being habitually late and regularly missing classes is disruptive and not fair to your classmates.  

\item \textit{Participation: }  Cellphone and computer usage in class for non-class related activities is strongly discouraged.  All devices should be set to silent when in class.  If your usage of technology becomes a distraction to your classmates or your instructor, then your participation grade will suffer.  If you're not sure if your being a distraction, then err on the side of caution and assume your distracting someone.  Put another way, if the instructor or a classmate has to tell you you're distracting them, then you've already gone too far. 

\item \textit{Quality of Work:} There are several minimal requirements that your assignments must meet.
\begin{itemize}
\item \textit{Electronic Submissions}  Most of your work will be handed in electronically.  It is your responsibility to know and understand the system for doing so and to be sure your work has properly submitted. Not following the instructions for assignment submission can mean your assignment does not get submitted and will be considered late. 

\item \textit{Staples - } Assignments that take up more than one page must be stapled.  Unstapled assignments will either be returned to you to be stabled ASAP or points will be deducted.  

\item \textit{Neatness - }  Make every attempt to make your work neat and orderly:  label problems, avoid excessive scratching out of mistakes (use pencil if you are prone to errors) and if you use spiral bound paper tear off the edges. Put your name on your work!

\item \textit{Show Work - } Rarely are answers alone sufficient for full credit.  Show your work whenever prudent.  If you're unsure if work is needed, \textit{ask!}
\end{itemize}

\end{itemize}


\subsection{Collaboration}

In general, you are encouraged to make use of the resources available to you.  This means it is OK to seek help from a friend, tutor, instructor, internet, etc.  However, \textit{copying of answers and any act worthy of the label of ``cheating'' is never permissible!}  It is understandable that when you work with a partner or a group that the resultant product is often extremely similar.  This is acceptable but be prepared to be asked to defend your collaborations to the instructor.  \textit{You should always be able to reproduce an answer on your own, and if you cannot you likely \textbf{do not really known the material.}} All of the Monmouth College rules on academic dishonesty apply.  If you violate the rules be prepared to face the consequences of your actions.  

\section{Grades}

This courses uses a standard grading scale.  Assignments and final grades will not be curved except in rare cases when its deemed necessary by the instruction.  Percentage grades translate to letter grades as follows:
\newline
\begin{small}
\begin{tabular}{lc}
94-100 & A \\
90-93 & A- \\
88-89 & B+ \\
82-87 & B \\
80-81 & B- \\
78-79 & C+ \\
72-77 & C \\
70-71 & C- \\
68-69 & D+ \\
62-67 & D \\
60-61 & D- \\
0-59 & F 
\end{tabular}
\end{small}
\newline
You are always welcome to challenge a grade that you feel is unfair or calculated incorrectly.  Mistakes made in your favor will never be corrected to lower your grade.  Mistakes made not in your favor will be corrected.  \textit{Basically, after the initial grading your score can only go up as the result of a challenge.}

\subsection{Grade Weights}
Your final grade is based on a weighted average of particular assignment categories.  You should be able to estimate your current grade based on your scores and these weights.  You may always visit the instructor \textit{outside of class time} to discuss your current standing.  
\begin{itemize}
\item Quizzes - 25\%
\item Projects - 25\%
\item Final - 12.5\%
\item Midterm - 12.5\%
\item Homework - 10\%
\item Labs - 10\%
\item Participation \& Attendance - 5\%
% add more if needed
\end{itemize} 

\subsection{Workload}
% number of/details on midterms, finals, project, homeworks, quizes, etc
The course workload is as follows:
\begin{itemize}
\item 10 Labs
\item 7 Homework Assignments
\item 5 Quizzes
\item 2 Projects
\item 1 Final Exam
\item 1 Midterm Exam
\end{itemize}

Homework assignments will generally be attached to labs as either a pre-lab or post-lab assignment. You can, therefore, expect them to be assigned in conjunction with the majority of the labs.

\subsection{Course Engagement Expectations}

The weekly workload for this course will vary by student but on average should be about 12.5 hours per week.  The follow tables provides a rough estimate of the distribution of this time over different course components for a 15 week semester. 
\begin{center}
\begin{tabular}{|l|l|l|}
\hline
Lectures+Labs+Final &      & 4.2 hours/week \\ 
Homework & 30 hours        & 2 hours/week \\
Exam Study Time & 8 hours  & 0.5 hours/week \\ 
Quiz Study Time & 10 hours & 0.6 hours/week \\
Projects & 48 hours        & 3.2 hours/week \\
Reading+Unstructured Study & & 2 hours/week \\
\hline 
& & 12.5 hours/week \\ 
\hline
\end{tabular}
\end{center}


\subsection{Calendar}

The following calendar should give you a feel for how work is distributed throughout the semester.  \textit{This calendar is subject to change based on the circumstances of the course.}
\begin{center}
\begin{tabular}{|c|c|r|}
\hline 
Week & Dates & Assignments \\
\hline
1 & 1/12 - 1/16 &  Lab 1.\\
\hline
2 & 1/19 - 1/23 & Lab 2.  Quiz 1.\\
\hline
3 & 1/26 - 1/30 & Lab 3.  \\
\hline
4 & 2/2 - 2/6 & Lab 4. Quiz 2.  \\
\hline
5 & 2/9 - 2/13 & Project 1 Homework. \\
\hline
6 & 2/16 - 2/20 & Project 1. \\
\hline
7 & 2/23 - 2/27 & Lab 5.  \\
\hline
8 & 3/2 - 3/6 & Lab 6. Midterm Exam.  \\
\hline 
SPRING BREAK & 3/9 - 3/13 &  \\
\hline
9 & 3/16 - 3/20 & Lab 7. \\
\hline
10 & 3/23 - 3/27 & Lab 8.  Quiz 3. \\
\hline
11 & 3/30 - 4/3 & Lab 9. EASTER BREAK (Friday).\\
\hline
12 & 4/6 - 4/10 & EASTER BREAK (Monday). Lab 10. Quiz 4. \\
\hline
13 & 4/13 - 4/17 &  . Project 2 Homework. \\
\hline
14 & 4/20 - 4/24 &  Project 2. \\
\hline
15 & 4/27 - 5/1 & Quiz 5. \\ 
\hline
16 & 5/4 - 5/6 & \\
\hline
Final's Week & 5/11 (6:30-9:30pm) & Final Exam. \\ 
\hline
\end{tabular}
\end{center}

\end{document}