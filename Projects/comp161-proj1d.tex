\documentclass[]{tufte-handout}
\usepackage{amsmath, amssymb, amsthm}
\usepackage{hyperref}
\usepackage{framed}
\usepackage{enumitem}
\usepackage{booktabs}

\title{COMP161 \\ Project 1 \\ Tic-Tac-Toe}
\author{}
\date{Spring 2019}

\begin{document}
\maketitle

\begin{abstract}
For your first project you'll be writing a simple C++ CLI implementation
of the game tic-tac-toe. The final product will allow for player-vs-player or
player-vs-computer games.  You will be asked to develop this program in a bottom-up
fashion by beginning with key elements of the game's procedure library and working
towards a running prototype only in the later versions of the program.
\end{abstract}

\section{Program Overview}

If you do not know how to play tic-tac-toe, ask just about anyone you know and
they can probably teach you the game.  Below you'll find some discussion about
representing the game board and identifying spots on the board as well as
a description of the versioning you should following in this program.

\subsection{Board as 1D or 2D vector}

You get to choose one of two representations of the board. The first is a
flat, 1D vector of size 9 and the second is a 3x3 2D vector, a.k.a. a vector of
vectors. We'll discuss the each approach in class but the basic observation
is that there are 9 spaces on a tic-tac-toe board and so long as we can translate
a 1D, zero to eight coordinate to a 2D (row,column) coordinate, and vice versa,
then we can choose a flat 1D vector or a grid-like 2D vector.  To keep things
uniform, we'll all use the following board location number scheme where
the top left spot is 0 and the bottom right spot is 8.

\begin{figure}
  \begin{tabular}{@{}|c|c|c|@{}}
    \midrule 0 & 1 & 2 \\
    \midrule 3 & 4 & 5 \\
    \midrule 6 & 7 & 8 \\
    \midrule
  \end{tabular}
  \caption{Tic-tac-toe board with locations numbered}
\end{figure}


\begin{figure}
  \begin{tabular}{@{}|c|c|c|c|c|c|c|c|c|@{}}
    \midrule
    0 & 1 & 2 & 3 & 4 & 5 & 6 & 7 & 8 \\
    \midrule
  \end{tabular}
  \caption{Tic-tac-toe board as 1D vector}
\end{figure}

Regardless of which way you choose to represent the board, it ends up being useful
to be able to translate between 1D and 2D coordinates and you'll be required to
do so as part of the project.


\subsection{Version 0}

The first version of your program is about developing procedures that visualize
the game state, test for win/loss conditions, and find the locations of all the
allowable moves. There is no running program here, just some very necessary
functionality for tic-tac-toe.

\begin{itemize}
  \item A procedure, or suite of procedures, to determine if the game has been
   won and by whom.
  \item Three procedures for converting board positions between 1D and 2D
    coordinates: 2D to 1D, 1D to 2D row coordinate, and 1D to 2D column coordinate.
  \item A procedure that produces a vector containing the 1D coordinates for
  all the currently allowable moves.
  \item A procedure that writes a visual representation of the board to the
  terminal.
\end{itemize}


\subsection{Version 1}

This version of the program adds procedures for getting and handling a move from
a player or, in the case of a random move, an AI.  After completing this version
of the program you still only have a library of helpful procedures, but it's
worth noting that these procedures more or less cover what you need to get the
full game moving.

\begin{itemize}
  \item A procedure, or series of procedures for getting a legal move from a
  player.
  \item A procedure to select a random legal move.
  \item A procedure to apply a given move, specified as a 1D coordinate, to
  the board.
\end{itemize}



\subsection{Version 2}

You will now take the library you've developed and implement a simple tic-tac-toe
program that allows the user to play one game against an AI opponent. Don't worry
about anything other than being able to play one complete game against the
computer.

\begin{itemize}
  \item A program that allows for a single PvAI game of tic-tac-toe. The AI goes
  first and uses the strategy of picking any allowable move at random. The program
  terminates as soon as the game is over.
\end{itemize}

\subsection{Version 3}

The final version of the program adds repeatable plays, the option to play against
other people, and a few other UI bells and whistles.

\begin{itemize}
  \item Modify the program so that it begins by asking if you'd like to play PvP
  or PvAI. When the game is over, announce the winner and ask the user if they'd
  like to play again. If they choose to keep playing, go back to asking what kind
  of game they want to play.
\end{itemize}

\subsection{Optional: Putting the I in AI}

Currently, your AI is not very intelligent. It simply picks a possible move at random
rather than attempting to determine which of those moves is best. Intelligence
implies at least some evaluation of move quality.

We can judge the quality of a move based on how good the board looks to the player
after they've made their move. Towards this end we can define a simple function
that assigns a numerical value to a given board configuration.
\begin{itemize}
  \item 200 if the AI has won
  \item 20 for each two in a row the AI has on the board
  \item 5 for each one in a row the AI has on the board
\end{itemize}
The AI now does the following: determine all possible moves, determine the value
of the board that results from each of those moves, and choose the move that
results in the board state with the highest value. Feel free to play around with
other ways of evaluating a board position. If you want to go all out on this then
look into the following algorithms: minimax, alpha-beta pruning, and A*-search.

\section{Grading}

Your grade is a function of the overall quality of the code you submit and how much of the
program you completed as measured by the version requirements listed above. Completeness
will determine the broad, letter-range of your grade while quality determines where
you fall within that range.

\subsection{Quality Points}

To earn full credit for code quality you must use good programming style, have well written documentation, and a full set of tests for all procedures. Further more, your design should be making good use of helper procedures in order to manage the complexity of the program. Cramming everything into a small number of procedures is likely to cause you to lose some points on design quality but you can earn a good amount of points if the procedures you have are well documented and tested using gTest. On the other hand, you can do an excellent job breaking things down into basic procedures but lose points to sparse or missing documentation and tests. Finally, you should make some effort at producing a decent user experience.  You don't need to go overboard one visual elements, but the
experience of playing your game should not be unpleasant or difficult to figure out.

\subsection{Correctness and Completeness Points}

Your overall grade range is based on the highest version number you complete as listed below. For example
if you complete all of the features listed in version 0 and 1 then you can expect
to get something in the C range based on the quality and correctness of what you've
submitted. In the event that you also completed some of the features in version 2, then
it is possible you'll get as high as a B- so long as your code is of sufficient quality and
correctly implements all the required features.  However, you cannot boost your grade by
means of cherry picking bits and pieces of each version.  Having only parts of version
0, 1, and two complete will be treated more or less like an incomplete version 0 project. 
Finally, above all else, be certain that your code compiles without error. Code that
does not compile will not receive a passing grade.


\begin{tabular}{@{}ll@{}}
\underline{Version} & \underline{Grade} \\
0 & D\\
1 & C\\
2 & B\\
3 & A\\
\end{tabular}


\newpage

\section{Timeline}

When it's due, submit your code as assignment \textit{proj1} using handin.
\begin{table}[!htpb]
  \begin{tabular}{ll}
    \underline{Date} & \underline{Assignment Due} \\
      3/27 & Begin Project in Lab \\
      4/3  & Work time in Lab \\
      4/9 & Project Due
  \end{tabular}
  \caption{Project Due Dates}\label{tab:duedates}
\end{table}

\end{document}
